\section{Introduction}
% a referenced introductory paragraph of approximately 150 words
The nuclear charge radii of neutron-deficient $^{177-185}$Hg isotopes have been investigated by in-source resonance ionization spectroscopy at ISOLDE, CERN.  The isotope shifts of the 253.72 nm atomic spectral line were measured using the Resonance Ionization Laser Ion Source (RILIS) operating in a reduced line-width scanning mode.  The RILIS was coupled for the first time to a liquid lead target, by employing a newly developed laser-ionization mode of the standard plasma ion source. These new measurements span the renowned region of nuclear shape-staggering, discovered at ISOLDE more than 40 years ago, revealing $^{181}$Hg to be its endpoint, with the ground-states of odd-N isotopes below N=100 re-joining the quasi-spherical trend observed in the even-N isotope chain.  The remarkable symmetry of the charge radii systematics either side of the neutron mid-shell is a strong indication of the magic neutron number N=82.
The results give a deep insight into the macroscopic picture of the charge radii systematics in this region.