\section{Abstract}
The nuclear charge radii of neutron-deficient 177-185Hg isotopes has been investigated by in-source resonance ionization spectroscopy at ISOLDE, CERN.  Optical isotope shifts were measured using the Resonance Ionization Laser Ion Source (RILIS), coupled for the first time to a liquid lead target, through the use of a newly-established RILIS-mode of operating the FEBIAD-type ion source.  combined with selective and high-sensitivity ion detection methods  and  using a highly-efficient three-step ionization scheme, and coupling it for the first time to a liquid-lead target through the use of the ISOLDE FEBIAD ion source.  using the Resonance Ionization Laser Ion Source (RILIS) at ISOLDE, CERN.  This work extends  the isotope shift measurements down to 

study concludes the renowned optical isotope-shift measurements, performed at ISOLDE and reported by Bonn et al. and Kuhl et al.

shape of exotic even-mass Pb182–190 isotopes was probed by measurement of optical isotope shifts providing mean square charge radii (δ⟨r2⟩). The experiment was carried out at the isolde (cern) on-line mass separator, using in-source laser spectroscopy. Small deviations from the spherical droplet model are observed, but when compared to model calculations, those are explained by high sensitivity of δ⟨r2⟩ to beyond mean-field correlations and small admixtures of intruder configurations in the ground state. The data support the predominantly spherical shape of the ground state of the proton-magic Z=82 lead isotopes near neutron midshell (N=104).


Isotope shifts and hyperfine structures have been measured on the 306.7 nm line in bismuth isotopes with A = 205-210, 210m, 212 and 213 by gas cell laser spectroscopy. More precise measurements were made for the A = 207-209 isotopes in atomic beam measurements. Nuclear magnetic and quadrupole moments were deduced. A detailed comparison of the nuclear charge radii systematics has been made in the region using a King plot technique