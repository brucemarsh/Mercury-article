\section{Main} 

% main text not to exceed 2,000 words; no more than 3–5 display items (figures, tables); references are limited to 30.

The neutron-deficient mercury isotopes were the subject of the first laser spectroscopy experiments on radioactive isotopes produced at an on-line isotope separator facility.  The experiment, which was conducted at the CERN SC-ISOLDE facility, revealed a dramatic jump in the ground state charge radius for the odd-mass isotopes between 181Hg and 185Hg [reference].  This unexpected result initiated a great deal of subsequent experimental theoretical activity, marking the beginning of a prolific, and still ongoing, investigation of nuclear ground state and isomer properties by laser spectroscopy in the lead region.  In a later experiment, using a technique known as 'RADOP', the ground-state properties of the neutron-deficient odd and even mass isotopes were studied.  This renowned work, reported by Bonn et al. and Kuhl et al. showed that the even isotopes do not share the large prolate ground state deformation of the odd-mass neighbors.  Instead, the isotopes chain exhibit the most pronounced odd-even shape staggering that has ever been observed.

In this work we report on the continuation of this stude, which, by using the most sensitive laser spectroscopy method now available - in-source resonance ionization spectroscopy - we have completed the picture of the Hg isotope chain down to 177Hg, which is the lightest Hg isotope ever observed.

shape of exotic even-mass Pb182–190 isotopes was probed by measurement of optical isotope shifts providing mean square charge radii (δ⟨r2⟩). The experiment was carried out at the isolde (cern) on-line mass separator, using in-source laser spectroscopy. Small deviations from the spherical droplet model are observed, but when compared to model calculations, those are explained by high sensitivity of δ⟨r2⟩ to beyond mean-field correlations and small admixtures of intruder configurations in the ground state. The data support the predominantly spherical shape of the ground state of the proton-magic Z=82 lead isotopes near neutron midshell (N=104).


Isotope shifts and hyperfine structures have been measured on the 306.7 nm line in bismuth isotopes with A = 205-210, 210m, 212 and 213 by gas cell laser spectroscopy. More precise measurements were made for the A = 207-209 isotopes in atomic beam measurements. Nuclear magnetic and quadrupole moments were deduced. A detailed comparison of the nuclear charge radii systematics has been made in the region using a King plot technique

The changes of the mean-square charge radii have been measured for 198Pt-183Pt by means of resonance ionization mass spectroscopy (RIMS) at the new on-line isotope separator ISOLDE-3/CERN. As in the case of the neighbouring isotopes of Au and Hg, a strong nuclear deformation of ¦β2¦ −-0.24 is reached at the neutron mid-shell nucleus183Pt, but no indication for a sharp shape transition is observed from the study of the isotope shifts.




The drastic jump between laTHg and laSHg has stimulated widespread experimental
and theoretical activity and, last but not least, also on-line laser spectroscopic
work on the even Hg isotopes and on the odd ila/2 isomers, which had not been
accessible in the RADOP experiments. These more recent results [12] have been
added in fig. 7, and now show the full truth of the story. The light Hg isotopes show a
very prominent odd-even shape staggering, including a shape isomerism in laSHg; the
even ones follow the trend of their heavier neighbours, being slightly oblate in shape.
This finding was already anticipated by 3" spectroscopic work in 184'la6Hg [13]. 

A satisfactory explanation of this puzzling effect was given by Pashkevich
and Frauendorf [14]. The potential energy surface has two almost degenerate minima,
one on the oblate side at/3 = - 0.1, i.e. almost spherical. This minimum is favoured
by the almost closed proton shell. The other minimum lies far out on the prolate side
at /3 = 0.25 and is favoured by the open neutron shell. The decision between both is
taken by the odd particle through its blocking effect on the neutron pairing interaction.
The neutron pairing is particularly strong on the oblate side, since there the
neutrons feel themselves in an unfavourable situation, that is at a very high density
of states near the Fermi surface which, on the other hand, favours pairing. This stabilization
of the oblate shape by the neutron pairing is lost, however, by the blocking of
an odd particle and hence the nucleus turns over into the prolate shape. 

% {The nuclei in the transitional
% region between the well-deformed isotopes of the rare
% earth elements and the doubly-magic 2~ (Z _< 82
% and N ~ 104) have attracted considerable attention during
% the last years following the observation of a sudden
% change of the nuclear deformation in the light Hg isotopes
% [1].


Meanwhile a shape transition has been observed
% in the light Au isotopes, too [2]. Since these effects
% critically depend on shell and pairing energies, it
% is interesting to study the behaviour of platinum, the
% lighter, neighbouring element of Au and Hg.

Over 30 years ago isotope shifts and hyperfine structures of mercury and gold isotopes were measured by laser spectroscopy at ISOLDE [1,2,3]. Nuclear charge radii, spins, and magnetic and quadrupole moments were determined, revealing the sudden and unforeseen onset of shape staggering for the Hg isotopes and an onset of deformation for the Au isotope chain.  These measurement sparked a great deal of interest in this transitional re
These effects are a consequence of shell and pairing energies and have been explained qualitively by Otten et al by the existence of two local minima in the deformation energy potential.

and sparking a great interest in this region of the nuclear chart.  These experiments were performed by either collinear florescence spectroscopy (Hg) or resonance ionization in an atomic beam after isotope collection (Au).  As such, the respective sensitivity and lifetime constraints imposed by the experimental conditions limited their reach towards more exotic isotopes.  Today, the Resonance Ionization Laser Ion Source (RILIS) [4], which the most commonly used ion source at ISOLDE, can also be operated in an enhanced resolution, wavelength-scanning mode. In this case the ionization efficiency is sensitive to the isotope shift or hyperfine structure during a laser scan.  This represents the most sensitive laser spectroscopy method at ISOLDE. Although the spectral resolution is limited by the Doppler broadening of atomic transitions inside the ionization region, for the heavier elements this is not prohibitive to the extraction of nuclear structure information.  By exploiting the array of ion beam detections setups that are available at ISOLDE: (α/β/γ) detection with the Leuven Windmill system; direct ion counting with the ISOLTRAP MR-TOF MS; and ion beam current measurements using the ISOLDE Faraday cups, our collaboration has performed extensive laser and nuclear-spectroscopic studies of isotopes in the lead region [5,6,7].  During 2015, two further studies were highly successful: Experiment IS534 [8] measured isotopes and isomers of 176-182Au for the first time, revealing that the Au ground states, which become prolate-deformed at N=108 [3], re-join the near-spherical trend below N=101, with a pronounced shape-staggering in the transition region; Experiment IS598 [9] measured 15 mercury isotopes, extending the charge radii systematics below N=101 down to 177Hg, and above N=126 up to 208Hg.  The results delineate the region of pronounced shape-staggering, and thereby provide a long-awaited conclusion to one of the earliest and most renowned measurements in this field.   Our recent results will be summarized and an outlook towards future studies and developments of the technique will be presented.   the last 


the