\section{Introduction} 

% {The nuclei in the transitional
% region between the well-deformed isotopes of the rare
% earth elements and the doubly-magic 2~ (Z _< 82
% and N ~ 104) have attracted considerable attention during
% the last years following the observation of a sudden
% change of the nuclear deformation in the light Hg isotopes
% [1]. Meanwhile a shape transition has been observed
% in the light Au isotopes, too [2]. Since these effects
% critically depend on shell and pairing energies, it
% is interesting to study the behaviour of platinum, the
% lighter, neighbouring element of Au and Hg.

Over 30 years ago isotope shifts and hyperfine structures of mercury and gold isotopes were measured by laser spectroscopy at ISOLDE [1,2,3]. Nuclear charge radii, spins, and magnetic and quadrupole moments were determined, revealing the sudden and unforeseen onset of shape staggering and deformation for the neutron-deficient isotopes, and sparking a great interest in this region of the nuclear chart.  These experiments were performed by either collinear florescence spectroscopy (Hg) or resonance ionization in an atomic beam after isotope collection (Au).  As such, the respective sensitivity and lifetime constraints imposed by the experimental conditions limited their reach towards more exotic isotopes.  Today, the Resonance Ionization Laser Ion Source (RILIS) [4], which the most commonly used ion source at ISOLDE, can also be operated in an enhanced resolution, wavelength-scanning mode. In this case the ionization efficiency is sensitive to the isotope shift or hyperfine structure during a laser scan.  This represents the most sensitive laser spectroscopy method at ISOLDE. Although the spectral resolution is limited by the Doppler broadening of atomic transitions inside the ionization region, for the heavier elements this is not prohibitive to the extraction of nuclear structure information.  By exploiting the array of ion beam detections setups that are available at ISOLDE: (α/β/γ) detection with the Leuven Windmill system; direct ion counting with the ISOLTRAP MR-TOF MS; and ion beam current measurements using the ISOLDE Faraday cups, our collaboration has performed extensive laser and nuclear-spectroscopic studies of isotopes in the lead region [5,6,7].  During 2015, two further studies were highly successful: Experiment IS534 [8] measured isotopes and isomers of 176-182Au for the first time, revealing that the Au ground states, which become prolate-deformed at N=108 [3], re-join the near-spherical trend below N=101, with a pronounced shape-staggering in the transition region; Experiment IS598 [9] measured 15 mercury isotopes, extending the charge radii systematics below N=101 down to 177Hg, and above N=126 up to 208Hg.  The results delineate the region of pronounced shape-staggering, and thereby provide a long-awaited conclusion to one of the earliest and most renowned measurements in this field.   Our recent results will be summarized and an outlook towards future studies and developments of the technique will be presented.